\chapter{nlmixr}

\section{Example 1}

\begin{table}[h]
    \centering
    \begin{tabular}{|l|l|}
        \hline
        \textbf{Variable} & \textbf{Description} \\ 
        \hline
        ID      & Subject ID \\ 
        TIME    & Time (hours) \\ 
        DV      & Drug concentration (mg/L) \\ 
        AMT     & Dose administered (mg) \\ 
        WT      & Weight of the subject (kg) \\ 
        EVID    & Event ID (e.g., dosing event vs. observation) \\ 
        CMT     & Compartment number (for PK models) \\ 
        \hline
    \end{tabular}
    \caption{Description of variables in the \texttt{theo\_sd} dataset.}
    \label{tab:theo_sd}
\end{table}

\section{Marie}
The ini block provides initial values, and is analogous to \$THETA, \$OMEGA and \$SIGMA blocks in NONMEM. The  model block specifies the model, and is analogous to the \$PK, \$PRED and \$ERROR blocks in NONMEM.

Solved PK systems are also currently supported by nlmixr with the ‘linCmt()’ pseudo-function. The solved systems implemented are the one, two and three compartment models with or without first-order absorption. In this example, linCmt() is used, and thus rxode2-based solved PK 1-compartment model with first-order absorption => nlmixr interprets this as a one-compartment model with depot.

The model for $\theta_i = (k_{a,i}, CL_{i}, V_i)^\top$ is
\begin{align*}
    k_{a,i}&=\exp(\beta_1+\eta_{i,k_a}),\\
    CL_i &= \exp(\beta_2+\eta_{i,CL})\\
    V_{i}&=\exp(\beta_3+\eta_{i,V}),\\
\end{align*}
where $\beta_1 = \log(k_a)$, $\beta_2 = \log(CL)$ and $\beta_3=\log(V)$. These are named tka, tcl and tv in lines 6-9. The matrix $\Omega$ is diagonal with elements being the values of eta. The initial values of this are the eta.ka, eta.cl and eta.v values specified in lines 11-13. 

The model $g(\cdot)$ is equivalent with \eqref{eq: conc. one-comp with abs, F=1}.

The residual error is in this example specified to be additive, i.e. $R_i = \sigma^2I$, hence we only need to specify initial values for the sd.

When the model is fitted, the condition values indicate whether the covariance matrix is close to singluar - i.e. if the parameters are strongly correlated.

The Time-lines provide a breakdown of the time taken during various stages of the model fitting process using nlmixr.

The Population Parameters are first of all the estimated values $\hat{\beta}$ and $\hat{\sigma}$ along with their SE. The $\%RSE$ is $RSE(\hat{x})=SE(\hat{x})/\hat{x}$, the relative SE. Back-transformed(95\%CI) are the variables on their original scale (not log-scale) along with CI. BSV is $CV=SD(\hat{x})/\Bar{\hat{x}}*100$. This indicates the relative variability of the parameter estimate among subjects. A higher CV\% suggests greater variability among individuals in terms of the parameter estimated. The shrinkage is a measure of how much the individual estimates (like individual clearance rates) have been pulled towards the population mean (low shrinkage = individual data contributed much).

\section{Mads}

Compartment $1$ only happens at dosing time, so compartment 1 might be the absorption site. We have 12 different individuals with 12 observations each. 

\begin{lstlisting}
    # The PK-model
one.cmt <- function() {
  ini({
    ## You may label each parameter with a comment
    tka <- 0.45 # Log Ka
    tcl <- log(c(0, 2.7, 100)) # Log Cl
    ## This works with interactive models
    ## You may also label the preceding line with label("label text")
    tv <- 3.45; label("log V")
    ## the label("Label name") works with all models
    eta.ka ~ 0.6
    eta.cl ~ 0.3
    eta.v ~ 0.1
    add.sd <- 0.7
  })
  model({
    ka <- exp(tka + eta.ka)
    cl <- exp(tcl + eta.cl)
    v <- exp(tv + eta.v)
    linCmt() ~ add(add.sd)
  })
}

f <- nlmixr(one.cmt)
\end{lstlisting}

Line $4-9$ are the fixed effects. Line $10-13$ are random effects. The sim used for the random effects are used to denote a variance for the random effects. 

Line $14$ is the residual error

Line $17-19$ is $\theta$ i.e. the model parameters. Note that they include a random effect and a fixed effect

Line $20$ is the residual variance model chosen. 

In this model no ODEs are specified. If this was included, it should have been a part of the model({})


For the fixed effects:
\begin{itemize}
    \item tka is the log absorption rate
    \item tcl is the elimination rate.
    \item tv is the log volume of distribution
\end{itemize}
tcl is a vector with $3$ entries. You can do that with all fixed effects, and in this case it means that the effect can vary from 0-100, but it should start at 2.7.

Remember eta is the greek letter.

In model({}) the parameters are specified as:
\begin{align*}
    K_{a}&=\exp(log(0.45)+\eta),  \quad \eta \sim 0.6\\
     Cl &= \exp(log(2.7) + \eta), \quad \eta \sim 0.3\\
    V_{d}&=\exp(log(3.45) + \eta), \quad \eta \sim 0.1\\
\end{align*}

When fitting the model, the nlmixr() function is used. 
\begin{lstlisting}
    nlmixr(
model,
data,
est = "saem",
saemControl(print=50,
nBurn=200, nEm=300),
tableControl(cwres=TRUE))
\end{lstlisting}
The first three inputs are self explanatory. 

Saemcontrol() takes the inputs: 
\begin{itemize}
  \item \texttt{seed} \hfill Random seed: (99)
  \item \texttt{nBurn} \hfill Number of iterations in the SA (burn-in) step: (200)
  \item \texttt{nEm} \hfill Number of iterations in the EM step: (300)
  \item \texttt{nmc} \hfill Number of Markov chains: (3)
  \item \texttt{atol} \hfill Absolute convergence tolerance: (1e-8)
  \item \texttt{print} \hfill Iterations to complete before printing to console: (1)
  \item \texttt{...} \hfill Additional arguments
\end{itemize}
i.e. its the specifications for the estimation method. If the estimation methods $est = "focei", "foce", "foi", "fo"$ were used, then $foceiControl()$ could have been used. 

The tablecontrol takes the inputs: 
\begin{itemize}
  \item \texttt{cwres} \hfill Boolean indicating if you need to calculate conditional weighted residuals (CWRES). This is on by default for FOCE(i) routines. This will also generate WRES, CPRED, and CRES.
  \item \texttt{npde} \hfill Calculate npde residuals (NPDE). This will also generate EPRED and ERES.
  \item \texttt{nsim} \hfill Number of simulations used for NPDE: (default 300)
  \item \texttt{ties} \hfill Boolean indicating if noise will be added to avoid ties in NPDE calculation: (TRUE)
  \item \texttt{Seed} \hfill Random seed to use for NPDE calculation: (1009)
\end{itemize}


\begin{lstlisting}
fit <- nlmixr(one.cmt, theo_sd, est="saem",
              control=list(print=0))

print(fit)
\end{lstlisting}

Breakdown of the print: 

First is the time:
\begin{lstlisting}
     Time (sec fit$time):

        covariance  saem table compress other
elapsed       0.01 41.27  0.04     0.02  3.61
\end{lstlisting}
This gives the time it took the program during the fitting process. 

Next are the population parameter
\begin{lstlisting}
    Population Parameters (fit$parFixed or fit$parFixedDf):

       Parameter  Est.     SE %RSE Back-transformed(95%CI) BSV(CV%) Shrink(SD)%
tka       Log Ka 0.453  0.195 43.1       1.57 (1.07, 2.31)     71.4    -0.445%
tcl       Log Cl  1.02 0.0843 8.29       2.76 (2.34, 3.26)     27.2      3.88%
tv         log V  3.45 0.0467 1.35       31.5 (28.8, 34.5)     13.9      10.2%
add.sd           0.695                               0.695
\end{lstlisting}
"Parameter" is the parameter being estimated. Est. is the estimated value. SE is the standard error. \%RSE is the relative standard error. Back-transformed is the transformation back to the original scale and the 95\% confidence interval. BSV is the between subject variability. Shrink ??




\section{Erfaringer}
\begin{itemize}
    \item eta omkring 15
    \item Mindst 10 observationer
    \item prop.sd på omkring 0.4. For lav prop.sd kan give problemer
\end{itemize}


