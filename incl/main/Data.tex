\chapter{Data}
\section{Introduction}
The PK models in this section will be fitted using NLMM with the nlmixr2 package \citep{nlmixr, nlmixrarticle}. To be able to fit a model, a model specification, a dataset, and an estimation method must be provided.

The model specification must include initial values and potential boundaries for $\beta$, residual error variance, $\sigma^2$, and for the entries in $\Omega$. The functional forms of the $p$ entries in $d$ from \eqref{eq: NLME Stage 2}, along with the error model, must be specified.
The functional form of \ref{eq: NLME Stage 1} is specified by solving a system of given ODEs describing the mass balance equations for the given compartments, e.g. \eqref{eq: first order kinetic of amount in one com without abs}, \eqref{eq: first order kinetic of amount in central com with abs com}, or \eqref{eq: 2-comp central}.

% data structure
The data require at least a unique subject identifier, timestamp of the observations, a dependent variable, an amount or rate of drug, and an event identifier to describe the observation \citep{nlmixrarticle}. 

% estimation method
The FOCEI, covered in \ref{sec: Estimation methods}, is used as the estimation method, however, other methods, e.g. FO, FOCE, and saem, are also supported. 


%output
% The output of the nlmixr function is metrics of goodness of fit, i.e. objective function, information criteria, log likelihood, and condition number, the run time of the fitting process, the subject specific parameters with standard errors, residual squared error, BSV, and shrinkage, the fitted omega matrix and lastly the fitted data 

% metrics of goodness of fit
% The run time to fit the model
% the population parameter estimates + residual error
% Omega matrix
% fitted values for $\beta$s, $\eta$s and $\e_i$s
\section{Introduction to data}
\section{Questions}
\begin{itemize}
    \item What is the point of fitting models on cp?
    \item Number of models - number of subjects, CP/DV
    \item Initial values
    \item Covariate model
    \item How much in depth good of fitness should we do for the models
    \item Should we expect more/new data or models?
\end{itemize}


\section{Model without iiv}
\subsection{10 sample profile}
\begin{table}
\centering\centering
\caption{No BSV - CP 10}
\centering
\fontsize{8}{10}\selectfont
\begin{tabular}[t]{lllllll}
\toprule
\textbf{Parameter} & \textbf{Est} & \textbf{SE} & \textbf{\%RSE} & \textbf{Back-transformed} & \textbf{BSV} & \textbf{Shrinkage}\\
\midrule
tka & -5.39 & 0.806 & 15 & 0.00456 (0.00094, 0.0222) &  & \\
\midrule
tq & -3.05 & 1.3 & 42.6 & 0.0473 (0.00371, 0.603) &  & \\
\midrule
tcl & -2.94 & 0.0613 & 2.09 & 0.0529 (0.0469, 0.0597) &  & \\
\midrule
tvc & -0.104 & 0.965 & 923 & 0.901 (0.136, 5.97) &  & \\
\midrule
tvp & 1.51 & 1.85 & 123 & 4.51 (0.119, 171) &  & \\
\midrule
prop.sd & 0.429 &  &  & 0.429 &  & \\
\bottomrule
\end{tabular}
\end{table}

\begin{table}
\centering\centering
\caption{No BSV - DV 10}
\centering
\fontsize{8}{10}\selectfont
\begin{tabular}[t]{lllllll}
\toprule
\textbf{Parameter} & \textbf{Est} & \textbf{SE} & \textbf{\%RSE} & \textbf{Back-transformed} & \textbf{BSV} & \textbf{Shrinkage}\\
\midrule
tka & -5.78 & 0.0662 & 1.15 & 0.00308 (0.0027, 0.00351) &  & \\
\midrule
tq & -6.23 & 6.37 & 102 & 0.00198 (7.53e-09, 518) &  & \\
\midrule
tcl & -2.96 & 0.25 & 8.43 & 0.0517 (0.0317, 0.0844) &  & \\
\midrule
tvc & -0.342 & 0.192 & 56.1 & 0.71 (0.488, 1.03) &  & \\
\midrule
tvp & 4.03 & 16.7 & 415 & 56.1 (3.41e-13, 9.22e+15) &  & \\
\midrule
prop.sd & 0.437 &  &  & 0.437 &  & \\
\bottomrule
\end{tabular}
\end{table}

\begin{itemize}
    \item DV vs. CP: Lavere sd for CP. 
    \item Betydeligt højere estimeret F, VC, og desuden generelt højere estimerede værdier for CP
    \item SE generelt lavere for DV
\end{itemize}

\subsection{25 sample profile}
\begin{table}
\centering\centering
\caption{No BSV - CP 25}
\centering
\fontsize{8}{10}\selectfont
\begin{tabular}[t]{lllllll}
\toprule
\textbf{Parameter} & \textbf{Est} & \textbf{SE} & \textbf{\%RSE} & \textbf{Back-transformed} & \textbf{BSV} & \textbf{Shrinkage}\\
\midrule
tka & -5.87 & 0.0755 & 1.29 & 0.00282 (0.00243, 0.00327) &  & \\
\midrule
tq & -4.82 & 2.28 & 47.3 & 0.00804 (9.21e-05, 0.702) &  & \\
\midrule
tcl & -3.04 & 0.352 & 11.6 & 0.0477 (0.0239, 0.095) &  & \\
\midrule
tvc & -0.367 & 0.151 & 41.1 & 0.693 (0.515, 0.931) &  & \\
\midrule
tvp & 5.1 & 2.85 & 55.9 & 164 (0.614, 4.4e+04) &  & \\
\midrule
prop.sd & 0.486 &  &  & 0.486 &  & \\
\bottomrule
\end{tabular}
\end{table}
\begin{itemize}
    \item DV10 vs. DV25: Higher sd for 25sp
\end{itemize}

\begin{table}
\centering\centering
\caption{No BSV - DV 25}
\centering
\fontsize{8}{10}\selectfont
\begin{tabular}[t]{lllllll}
\toprule
\textbf{Parameter} & \textbf{Est} & \textbf{SE} & \textbf{\%RSE} & \textbf{Back-transformed} & \textbf{BSV} & \textbf{Shrinkage}\\
\midrule
tka & -5.9 & 0.0762 & 1.29 & 0.00273 (0.00235, 0.00317) &  & \\
\midrule
tq & -4.22 & 1.81 & 43 & 0.0147 (0.00042, 0.515) &  & \\
\midrule
tcl & -3.18 & 0.647 & 20.3 & 0.0414 (0.0116, 0.147) &  & \\
\midrule
tvc & -0.429 & 0.153 & 35.6 & 0.651 (0.483, 0.878) &  & \\
\midrule
tvp & 5.89 & 5.43 & 92.3 & 361 (0.00857, 1.52e+07) &  & \\
\midrule
prop.sd & 0.501 &  &  & 0.501 &  & \\
\bottomrule
\end{tabular}
\end{table}


\subsection{50 sample profile}
\begin{table}
\centering\centering
\caption{No BSV - CP 50}
\centering
\fontsize{8}{10}\selectfont
\begin{tabular}[t]{lllllll}
\toprule
\textbf{Parameter} & \textbf{Est} & \textbf{SE} & \textbf{\%RSE} & \textbf{Back-transformed} & \textbf{BSV} & \textbf{Shrinkage}\\
\midrule
tka & -5.88 & 0.0419 & 0.713 & 0.0028 (0.00258, 0.00304) &  & \\
\midrule
tq & -4.81 & 1.17 & 24.3 & 0.00815 (0.000821, 0.0809) &  & \\
\midrule
tcl & -3.02 & 0.198 & 6.55 & 0.0488 (0.0331, 0.0719) &  & \\
\midrule
tvc & -0.352 & 0.0959 & 27.2 & 0.703 (0.583, 0.849) &  & \\
\midrule
tvp & 5.48 & 4.22 & 76.9 & 241 (0.0619, 9.35e+05) &  & \\
\midrule
prop.sd & 0.485 &  &  & 0.485 &  & \\
\bottomrule
\end{tabular}
\end{table}

\begin{table}
\centering\centering
\caption{No BSV - DV 50}
\centering
\fontsize{8}{10}\selectfont
\begin{tabular}[t]{lllllll}
\toprule
\textbf{Parameter} & \textbf{Est} & \textbf{SE} & \textbf{\%RSE} & \textbf{Back-transformed} & \textbf{BSV} & \textbf{Shrinkage}\\
\midrule
tka & -5.91 & 0.0658 & 1.11 & 0.00272 (0.00239, 0.0031) &  & \\
\midrule
tq & -4.36 & 2.18 & 50 & 0.0127 (0.000177, 0.916) &  & \\
\midrule
tcl & -3.09 & 0.596 & 19.3 & 0.0456 (0.0142, 0.147) &  & \\
\midrule
tvc & -0.369 & 0.115 & 31.1 & 0.691 (0.552, 0.866) &  & \\
\midrule
tvp & 4.62 & 4.22 & 91.3 & 102 (0.0261, 3.95e+05) &  & \\
\midrule
prop.sd & 0.51 &  &  & 0.51 &  & \\
\bottomrule
\end{tabular}
\end{table}


\subsection{100 sample profile}
\begin{table}
\centering\centering
\caption{No BSV - CP 100}
\centering
\fontsize{8}{10}\selectfont
\begin{tabular}[t]{lllllll}
\toprule
\textbf{Parameter} & \textbf{Est} & \textbf{SE} & \textbf{\%RSE} & \textbf{Back-transformed} & \textbf{BSV} & \textbf{Shrinkage}\\
\midrule
tka & -5.92 & 0.0395 & 0.667 & 0.00268 (0.00248, 0.0029) &  & \\
\midrule
tq & -4 & 0.446 & 11.1 & 0.0183 (0.00761, 0.0438) &  & \\
\midrule
tcl & -3.3 & 0.226 & 6.84 & 0.037 (0.0238, 0.0575) &  & \\
\midrule
tvc & -0.413 & 0.075 & 18.2 & 0.661 (0.571, 0.766) &  & \\
\midrule
tvp & 6.19 & 2.16 & 35 & 485 (6.98, 3.37e+04) &  & \\
\midrule
prop.sd & 0.486 &  &  & 0.486 &  & \\
\bottomrule
\end{tabular}
\end{table}

\begin{table}
\centering\centering
\caption{No BSV - DV 100}
\centering
\fontsize{8}{10}\selectfont
\begin{tabular}[t]{lllllll}
\toprule
\textbf{Parameter} & \textbf{Est} & \textbf{SE} & \textbf{\%RSE} & \textbf{Back-transformed} & \textbf{BSV} & \textbf{Shrinkage}\\
\midrule
tka & -5.92 & 0.0529 & 0.893 & 0.00267 (0.00241, 0.00296) &  & \\
\midrule
tq & -3.69 & 0.248 & 6.71 & 0.0249 (0.0153, 0.0405) &  & \\
\midrule
tcl & -3.49 & 0.203 & 5.83 & 0.0305 (0.0204, 0.0454) &  & \\
\midrule
tvc & -0.423 & 0.0844 & 20 & 0.655 (0.555, 0.773) &  & \\
\midrule
tvp & 6.02 & 1.47 & 24.4 & 412 (23.1, 7.35e+03) &  & \\
\midrule
prop.sd & 0.49 &  &  & 0.49 &  & \\
\bottomrule
\end{tabular}
\end{table}


\section{Model with ETAs on CL and Q}
\subsection{10 sample profile}
\subsection{25 sample profile}
\subsection{50 sample profile}
\subsection{100 sample profile}

\section{Model with ETAs on V2 and V3}
\subsection{10 sample profile}
\subsection{25 sample profile}
\subsection{50 sample profile}
\subsection{100 sample profile}

\section{Model with ETAs on all}
\subsection{10 sample profile}
\subsection{25 sample profile}
\subsection{50 sample profile}
\subsection{100 sample profile}


\section{Model with BW as covariate and ETAs on CL and Q}
\subsection{10 sample profile}
\subsection{25 sample profile}
\subsection{50 sample profile}
\subsection{100 sample profile}


\section{Model with BW as covariate and ETAs on V2 and V3}
\subsection{10 sample profile}
\subsection{25 sample profile}
\subsection{50 sample profile}
\subsection{100 sample profile}


\section{Model with BW as covariate and ETAs on all}
\subsection{10 sample profile}
\subsection{25 sample profile}
\subsection{50 sample profile}
\subsection{100 sample profile}